% =========================
% PREAMBLE — JardinSolidaire
% =========================

% --- Langue ---
\usepackage[french]{babel}

% --- Mise en page ---
\usepackage[tmargin=1.6cm,bmargin=2cm,lmargin=1in,rmargin=1in,footskip=.2in]{geometry}
\usepackage{setspace}
\setstretch{1.08}
\usepackage{microtype}
\usepackage{parskip}

% --- Police ---
\usepackage{newpxtext}

% --- Images / flottants / listes ---
\usepackage{graphicx}
\usepackage{float}
\usepackage{enumitem}
\setlist[itemize]{noitemsep, topsep=4pt}
\setlist[enumerate]{noitemsep, topsep=4pt}

% Mets tous tes paths ici, en une seule fois (sinon tu écrases)
\graphicspath{{assets/cover/}{assets/}{assets/}{assets/17_annexe/}}

% --- Titrage chapitres ---
\usepackage{titlesec}
\titlespacing*{\chapter}{0pt}{-35pt}{16pt}

% --- Sommaire (tocloft AVANT hyperref) ---
\usepackage{tocloft}
\setcounter{tocdepth}{0}
\setlength{\cftbeforechapskip}{8pt}
\setlength{\cftbeforesecskip}{2pt}
\renewcommand{\cftchapfont}{\bfseries}
\renewcommand{\cftchappagefont}{\bfseries}
\renewcommand{\cftsecfont}{}
\renewcommand{\cftsecpagefont}{}
\renewcommand{\cftdotsep}{1.5}
\setlength{\cftchapindent}{0pt}
\setlength{\cftchapnumwidth}{2.2em}
\setlength{\cftsecindent}{2.2em}
\setlength{\cftsecnumwidth}{3.0em}

% --- Couleurs + liens ---
\usepackage{xcolor}
\definecolor{doc}{RGB}{0,60,110}

\usepackage{xurl}

\usepackage{hyperref}
\hypersetup{
  pdftitle={Projet Jardin Solidaire},
  colorlinks=true,
  linkcolor=doc!90,
  bookmarksnumbered=true,
  bookmarksopen=true
}
\usepackage{bookmark}

% --- Captions ---
\usepackage{caption}
\captionsetup{
  labelfont=bf,
  font=small,
  justification=centering,
  singlelinecheck=false
}

% --- Annexes PDF ---
\usepackage{pdfpages}

% --- TikZ ---
\usepackage{tikz}
\usetikzlibrary{positioning,arrows.meta}

% --- Libellés FR ---
\renewcommand{\chaptername}{Chapitre}
\renewcommand{\contentsname}{Sommaire}
\renewcommand{\listfigurename}{Liste des figures}
\renewcommand{\listtablename}{Liste des tableaux}
\renewcommand{\appendixname}{Annexes}

% --- Macro Annexe ---
\newcommand{\Annexe}[2]{%
  \clearpage
  \chapter*{Annexe #1 — #2}%
  \addcontentsline{toc}{chapter}{Annexe #1 — #2}%
}

% --- Espacements réutilisables ---
\newcommand{\dbspacebefore}{0.25cm}
\newcommand{\dbspaceafter}{0.15cm}

% --- Mode rapide pour Overleaf free ---
\newif\iffast
\fasttrue % mets \fastfalse pour la version finale

\graphicspath{{./}{./assets/}{./assets/}}

\usepackage{tabularx}
\usepackage{array}

\newcolumntype{Y}{>{\raggedright\arraybackslash}X}

% --- Boîtes 3x2 (sans tcolorbox) ---
\usepackage{xcolor}

% Couleurs (modifie si tu veux)
\definecolor{jsBoxBorder}{HTML}{2F3A34}
\definecolor{jsBoxBg}{HTML}{F4F7F5}

% Réglages boîtes
\newlength{\JSBoxSep}
\setlength{\JSBoxSep}{10pt} % espace horizontal entre boîtes

\newlength{\JSBoxPad}
\setlength{\JSBoxPad}{6pt}  % padding interne

\newlength{\JSBoxRule}
\setlength{\JSBoxRule}{0.6pt} % épaisseur bordure

\newlength{\JSBoxW}        % largeur d’une boîte (calculée)
\newlength{\JSBoxFullH}    % hauteur max de la ligne (calculée)
\newlength{\JSBoxInnerH}   % hauteur interne à appliquer (calculée)

\newsavebox{\JSBoxA}
\newsavebox{\JSBoxB}
\newsavebox{\JSBoxC}

% Police interne (plus petite)
\newcommand{\JSBoxFont}{\fontsize{8.2}{10}\selectfont}

% Boîte "brute" (hauteur auto) -> sert pour mesurer
\newcommand{\JSBoxRaw}[2]{%
  \begingroup
  \setlength{\fboxsep}{\JSBoxPad}%
  \setlength{\fboxrule}{\JSBoxRule}%
  \fcolorbox{jsBoxBorder}{jsBoxBg}{%
    \parbox[t]{\dimexpr\JSBoxW-2\fboxsep-2\fboxrule\relax}{%
      \JSBoxFont\textbf{#1}\par\vspace{4pt}#2%
    }%
  }%
  \endgroup
}

% Boîte "fixée" (hauteur imposée) -> sert pour afficher
\newcommand{\JSBoxFixed}[3]{%
  \begingroup
  \setlength{\fboxsep}{\JSBoxPad}%
  \setlength{\fboxrule}{\JSBoxRule}%
  \fcolorbox{jsBoxBorder}{jsBoxBg}{%
    \parbox[t][#3][t]{\dimexpr\JSBoxW-2\fboxsep-2\fboxrule\relax}{%
      \JSBoxFont\textbf{#1}\par\vspace{4pt}#2%
    }%
  }%
  \endgroup
}

% Une ligne de 3 boîtes (hauteurs automatiquement harmonisées)
% Usage : \JSRow{T1}{Txt1}{T2}{Txt2}{T3}{Txt3}
\newcommand{\JSRow}[6]{%
  % calc largeur d’une boîte = (textwidth - 2 espaces)/3
  \setlength{\JSBoxW}{\dimexpr(\textwidth-2\JSBoxSep)/3\relax}%

  % mesure des 3 boîtes en hauteur "naturelle"
  \sbox{\JSBoxA}{\JSBoxRaw{#1}{#2}}%
  \sbox{\JSBoxB}{\JSBoxRaw{#3}{#4}}%
  \sbox{\JSBoxC}{\JSBoxRaw{#5}{#6}}%

  % calc hauteur max (full height = ht + dp)
  \setlength{\JSBoxFullH}{\dimexpr\ht\JSBoxA+\dp\JSBoxA\relax}%
  \ifdim\dimexpr\ht\JSBoxB+\dp\JSBoxB\relax > \JSBoxFullH
    \setlength{\JSBoxFullH}{\dimexpr\ht\JSBoxB+\dp\JSBoxB\relax}%
  \fi
  \ifdim\dimexpr\ht\JSBoxC+\dp\JSBoxC\relax > \JSBoxFullH
    \setlength{\JSBoxFullH}{\dimexpr\ht\JSBoxC+\dp\JSBoxC\relax}%
  \fi

  % hauteur interne = hauteur max - padding/bordures (haut+bas)
  \setlength{\JSBoxInnerH}{\dimexpr\JSBoxFullH-2\JSBoxPad-2\JSBoxRule\relax}%

  % affichage des 3 boîtes avec la même hauteur
  \noindent
  \JSBoxFixed{#1}{#2}{\JSBoxInnerH}\hspace{\JSBoxSep}%
  \JSBoxFixed{#3}{#4}{\JSBoxInnerH}\hspace{\JSBoxSep}%
  \JSBoxFixed{#5}{#6}{\JSBoxInnerH}%
}
\usepackage{tabularx}
\usepackage{array}
% Colonnes p{} sans justification (évite les grands espaces)
\newcolumntype{P}[1]{>{\raggedright\arraybackslash}p{#1}}
\newcolumntype{Y}{>{\raggedright\arraybackslash}X}
