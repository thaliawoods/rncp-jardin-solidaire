\section*{Presentation (EN)}
\addcontentsline{toc}{section}{Presentation (EN)}

After several years working in restaurants, retail, and the social sector, and having long been drawn to \textbf{web development}, I decided in 2023 to change careers and train in this field.

I joined \textbf{Ada Tech School}, a two-part program: a 9-month full-time module at school, followed by a 12-month apprenticeship, leading to the RNCP Level 6 qualification \textbf{Application Designer and Developer}. I completed my apprenticeship at \textbf{Julaya}, a B2B fintech company, working in production with a modern stack (Next.js, TypeScript, PostgreSQL) in an agile environment.

This experience taught me how to build in a real product context: iterate step by step, integrate feedback, handle edge cases, and deliver \textbf{maintainable} code.

In this context, I developed \textbf{JardinSolidaire}.

I started from a simple observation: green spaces improve well-being and social connection, yet many private gardens remain underused. At the same time, many people want to garden, learn, breathe, and reconnect with nature, without having access to a garden, especially in cities. I wanted to connect these two realities through a \textbf{practical} and \textbf{simple} product.

\textbf{JardinSolidaire} is a mutual-aid web platform that connects \textbf{garden owners} with \textbf{volunteer gardeners}. The goal is not to build a commercial service: there is no payment. It focuses on shared time, cooperation, and knowledge transmission, bringing \og sleeping \fg{} gardens back to life and enabling local connections.

The project was initially started by three classmates from my cohort. Halfway through, we split up to each deliver a complete version, while keeping the same original intention.

At its current stage, JardinSolidaire allows users to create an owner or gardener profile, provide useful information (photos, location, needs or skills), browse a garden listing with filters and favorites, access clear detail pages, and organize a visit through time-slot scheduling with a calendar and a messaging system tied to the exchange.

JardinSolidaire sits at the intersection of \textbf{tech}, \textbf{social impact}, and \textbf{ecology}: a simple way to build connections, share skills, and take care of neighborhood green spaces.
