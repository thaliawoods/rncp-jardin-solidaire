\chapter{Limites, améliorations et perspectives}

Aujourd’hui, JardinSolidaire est un MVP \textbf{testable} : on peut créer un compte, publier ou consulter des jardins, proposer des créneaux, réserver et échanger. Ce socle me permet déjà de confronter le produit à un usage réel. Cependant il a encore des limites : il fonctionne, sans encore être optimisé pour l’expérience en conditions réelles.

Le premier enjeu, est de passer d’un simple \og catalogue  à une recherche vraiment \textbf{pertinente}. Pour l’instant, on peut trouver des jardins, mais pas forcément \textbf{celui qui nous convient} rapidement. Or, dans la réalité, tout se joue sur des critères très concrets : la distance, les disponibilités, le besoin précis, et les conditions sur place. La priorité, pour moi, est donc de rendre la recherche plus efficace avec de la \textbf{géolocalisation} et des \textbf{filtres utiles} (durée estimée, outils disponibles, point d’eau, accessibilité, niveau d’autonomie). L’objectif est simple : réduire le temps passé à ouvrir des fiches pour finalement se rendre compte que ce n’est pas adapté.

Une fois qu’on trouve un jardin pertinent, il faut que l’expérience soit \textbf{rassurante}. Techniquement, l’application est sécurisée (authentification, autorisations, validations), mais JardinSolidaire touche à un contexte sensible : on accueille quelqu’un chez soi, ou on se rend chez un inconnu. Pour que l’usage soit réellement fluide, je dois ajouter des mécanismes \textbf{côté usage} : avis, signalement et modération. Pas pour "surveiller", mais pour poser un cadre clair, de confiance, et éviter les zones grises qui freinent les utilisateurs.

Ensuite, je veux que cette expérience reste \textbf{accessible} au plus grand nombre. J’ai déjà posé des bases (labels, focus visible, messages clairs), mais je veux aller plus loin : renforcer la navigation clavier, vérifier les contrastes et éviter que des interactions trop visuelles (carte, hover) bloquent une partie des utilisateurs. JardinSolidaire vise l’inclusion : l’interface doit rester simple et utilisable pour des personnes très différentes.

Enfin, je pourrai ajouter ce qui fait une partie de l’identité du projet : la \textbf{banque du temps}. Je la vois comme une manière de valoriser l’engagement sans monétiser le service, mais je veux la concevoir avec prudence, pour éviter l’effet "score" ou les abus, et rester dans une logique d’encouragement plutôt que de performance.

Les prochaines étapes sont donc claires : \textbf{faciliter la trouvaille}, \textbf{mieux rassurer}, \textbf{mieux inclure}, puis introduire la banque du temps sans dénaturer l’entraide.
