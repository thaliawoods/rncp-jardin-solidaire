\chapter{Présentation du projet}

\section*{Présentation du projet (FR)}
\addcontentsline{toc}{section}{Présentation du projet (FR)}

Après plusieurs années en restauration, en vente et dans le social, et attirée depuis longtemps par le \textbf{développement web}, j’ai choisi en 2023 de me réorienter et de me former à ce métier.

J’ai intégré \textbf{Ada Tech School}, une formation en deux temps : 9 mois à l’école, puis 12 mois en alternance, menant au titre RNCP niveau 6 \textbf{Concepteur·rice Développeur·se d’Applications}. J’ai réalisé mon alternance chez \textbf{Julaya}, une fintech B2B, dans un environnement de production avec une stack moderne (Next.js, TypeScript, PostgreSQL) et une organisation agile.

Cette expérience m’a appris à travailler sur un produit réel : avancer par itérations, intégrer les retours, traiter les cas limites, et livrer du code \textbf{maintenable}.

C’est dans ce cadre que j’ai développé \textbf{JardinSolidaire}.

À l’origine, je suis partie d’un constat simple : les espaces verts améliorent le bien-être et le lien social, mais beaucoup de jardins privés restent sous-utilisés. En parallèle, de nombreuses personnes aimeraient jardiner, apprendre, respirer, se reconnecter au vivant, sans forcément avoir accès à un jardin, surtout en ville. J’ai voulu construire une solution qui relie ces deux réalités, avec une approche \textbf{concrète} et \textbf{simple}.

\textbf{JardinSolidaire} est une plateforme web d’entraide qui met en relation des \textbf{propriétaires de jardins} et des \textbf{jardinier.es volontaires}. L’idée n’est pas de créer un service marchand : il n’y a pas de paiement. Je mise sur le temps partagé, la coopération et la transmission, pour remettre en vie des jardins qui \og dorment \fg{} et faciliter des rencontres de proximité.

Le projet a été initié à trois avec des camarades de ma promotion. À mi-parcours, nous avons choisi de nous séparer pour mener chacune une version aboutie, tout en gardant la même intention de départ.

Aujourd’hui, JardinSolidaire permet de créer un profil propriétaire ou jardinier·e, de renseigner des informations utiles (photos, localisation, besoins ou compétences), de consulter un listing de jardins avec filtres et favoris, d’accéder à des fiches détaillées, et d’organiser une intervention grâce à des créneaux planifiés via un calendrier et une messagerie liée à l’échange.

JardinSolidaire se situe au croisement de la \textbf{tech}, du \textbf{social} et de l’\textbf{écologie} : une solution simple pour créer du lien, partager des gestes, et prendre soin des espaces verts de quartier.
\newpage
\section*{Presentation (EN)}
\addcontentsline{toc}{section}{Presentation (EN)}

After several years working in restaurants, retail, and the social sector, and having long been drawn to \textbf{web development}, I decided in 2023 to change careers and train in this field.

I joined \textbf{Ada Tech School}, a two-part program: a 9-month full-time module at school, followed by a 12-month apprenticeship, leading to the RNCP Level 6 qualification \textbf{Application Designer and Developer}. I completed my apprenticeship at \textbf{Julaya}, a B2B fintech company, working in production with a modern stack (Next.js, TypeScript, PostgreSQL) in an agile environment.

This experience taught me how to build in a real product context: iterate step by step, integrate feedback, handle edge cases, and deliver \textbf{maintainable} code.

In this context, I developed \textbf{JardinSolidaire}.

I started from a simple observation: green spaces improve well-being and social connection, yet many private gardens remain underused. At the same time, many people want to garden, learn, breathe, and reconnect with nature, without having access to a garden, especially in cities. I wanted to connect these two realities through a \textbf{practical} and \textbf{simple} product.

\textbf{JardinSolidaire} is a mutual-aid web platform that connects \textbf{garden owners} with \textbf{volunteer gardeners}. The goal is not to build a commercial service: there is no payment. It focuses on shared time, cooperation, and knowledge transmission, bringing \og sleeping \fg{} gardens back to life and enabling local connections.

The project was initially started by three classmates from my cohort. Halfway through, we split up to each deliver a complete version, while keeping the same original intention.

At its current stage, JardinSolidaire allows users to create an owner or gardener profile, provide useful information (photos, location, needs or skills), browse a garden listing with filters and favorites, access clear detail pages, and organize a visit through time-slot scheduling with a calendar and a messaging system tied to the exchange.

JardinSolidaire sits at the intersection of \textbf{tech}, \textbf{social impact}, and \textbf{ecology}: a simple way to build connections, share skills, and take care of neighborhood green spaces.
