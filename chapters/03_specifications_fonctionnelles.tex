\chapter{Spécifications fonctionnelles}

% ---------------------------
% Expression des besoins
% ---------------------------
\section*{Expression des besoins}
\addcontentsline{toc}{section}{Expression des besoins}

JardinSolidaire vise à résoudre un besoin très concret : aujourd’hui, beaucoup de personnes possèdent un jardin mais n’ont pas toujours le temps, l’énergie ou les ressources pour l’entretenir.

En parallèle, beaucoup d’autres personnes aimeraient jardiner, apprendre, respirer et se reconnecter à la nature, mais n’y ont pas accès, surtout en ville.

L’objectif de l’application est donc de faciliter la \textbf{mise en relation} et l’\textbf{organisation} entre deux publics : les \textbf{propriétaires de jardins} et les \textbf{jardinier.es volontaires}.

J’ai choisi une interface web \textbf{responsive} pour que le service soit utilisable dans la vraie vie, sur ordinateur comme sur téléphone, sans installation. L’application doit couvrir aussi bien un coup de main ponctuel qu’un échange régulier (suivi saisonnier, rendez-vous récurrents), sans complexifier le parcours.

Enfin, j’ai fait le choix d’un modèle \textbf{non marchand} : pas de paiement, pour rester cohérente avec l’intention du projet. L’échange repose sur le temps partagé, l’entraide et la transmission. L’enjeu principal n’est donc pas d’optimiser un prix, mais de créer un cadre clair, rassurant et efficace pour rendre l’entraide possible.
\vspace{0.5cm}
\begin{figure}[H]
  \centering
  \includegraphics[width=\textwidth]{assets/03_specifications_fonctionnelles/expression_des_besoins.png}
  \caption{Synthèse du besoin auquel répond JardinSolidaire.}
  \label{fig:js-besoin-synthese}
\end{figure}
\newpage


% ---------------------------
% Démarche de cadrage
% ---------------------------
\section*{Démarche de cadrage : questionnaire, PESTEL, benchmark}
\addcontentsline{toc}{section}{Démarche de cadrage : questionnaire, PESTEL, benchmark}

Avant de construire l’application, j’ai cadré le projet pour vérifier qu’il répondait à des attentes réelles. Pour cela, j’ai commencé par un \textbf{questionnaire}.

Le questionnaire a été envoyé autour de moi pour identifier la cible, mais aussi pour comprendre les freins et les attentes des futur.es utilisateur.ice.s.

J’ai fait attention à l’\textbf{accessibilité} des questions : ici, "accessibilité" signifie surtout un langage clair, sans jargon, afin de toucher aussi des personnes moins à l’aise avec le numérique.

Le formulaire comportait une section "propriétaire" (fréquence d’entretien, disponibilités, types de besoins) et une section "jardinier.e" (niveau d’expérience, créneaux, motivations, contraintes).

J’ai reçu \textbf{27 réponses}, et les conclusions étaient très nettes : les utilisateur.ice.s attendent d’abord de la \textbf{simplicité} (fiches claires, consignes précises), ensuite de la \textbf{visibilité} (créneaux lisibles sur un calendrier), et enfin de la \textbf{confiance} (profils détaillés, messagerie, et à terme avis/notes).

\subsection*{Annexe A — Questionnaire complet}
\addcontentsline{toc}{subsection}{Annexe A — Questionnaire complet}

\subsection*{Annexe B — Résultats détaillés du questionnaire}
\addcontentsline{toc}{subsection}{Annexe B — Résultats détaillés du questionnaire}

\vspace{0.5cm}

\subsection*{Analyse PESTEL}
\addcontentsline{toc}{subsection}{Analyse PESTEL}

En complément, j’ai réalisé une \textbf{analyse PESTEL}. C’est une méthode qui permet d’analyser un projet dans son contexte global en regardant six dimensions : \textbf{Politique}, \textbf{Économique}, \textbf{Sociale}, \textbf{Technologique}, \textbf{Environnementale} et \textbf{Légale}.

L’intérêt est de repérer à la fois les opportunités (par exemple le regain d’intérêt pour le jardinage ou les initiatives d’espaces verts) et les contraintes (par exemple la sécurité des personnes, la protection des données et le cadre légal).


\newcommand{\pestelBoxW}{0.31\textwidth} 
\newcommand{\pestelGapV}{2pt}           
\newcommand{\pestelInnerPad}{2pt}      

\newcommand{\pestelBullet}{\(\circ\)\hspace{0.3em}}

\newcommand{\pestelBox}[2]{%
  \fbox{%
    \begin{minipage}[t][0.19\textheight][t]{\pestelBoxW}
      \vspace{\pestelInnerPad}

      \textbf{#1}\par\vspace{1pt}

      {\footnotesize
        \setlength{\parskip}{1pt}
        \setlength{\baselineskip}{10pt}
        #2
      }

      \vspace{\pestelInnerPad}
    \end{minipage}%
  }%
}

\begin{figure}[H]
\centering

\pestelBox{P — Politique}{%
Opportunités :\par
\pestelBullet Initiatives publiques autour du verdissement et du lien social.\par
\pestelBullet Possibles partenariats locaux (assos, quartiers).\par
\vspace{4pt}
Contraintes :\par
\pestelBullet Clarifier le rôle de la plateforme en cas d’incident.\par
\pestelBullet Définir un cadre en cas de partenariat (responsabilités).%
}
\hfill
\pestelBox{E — Économique}{%
Opportunités :\par
\pestelBullet Alternative à l’entretien payant.\par
\pestelBullet Valorise l’échange de temps (entraide).\par
\vspace{4pt}
Contraintes :\par
\pestelBullet Modèle non marchand : valeur à rendre très claire.\par
\pestelBullet Besoin d’un parcours simple et efficace (sinon abandon).%
}
\hfill
\pestelBox{S — Social}{%
Opportunités :\par
\pestelBullet Réduit l’isolement, crée du lien de proximité.\par
\pestelBullet Accès au vivant pour des personnes sans jardin.\par
\vspace{4pt}
Contraintes :\par
\pestelBullet Confiance : peur d’accueillir un inconnu.\par
\pestelBullet Besoin de consignes claires et d’un cadre rassurant.%
}

\vspace{\pestelGapV}

\pestelBox{T — Technologique}{%
Opportunités :\par
\pestelBullet Web mobile-first : accessible sans installer d’app.\par
\pestelBullet Carte + calendrier + messagerie : organisation facilitée.\par
\vspace{4pt}
Contraintes :\par
\pestelBullet Sécurité (auth, données) et fiabilité des réservations.\par
\pestelBullet Prévention des abus (signalement / modération).%
}
\hfill
\pestelBox{E — Environnemental}{%
Opportunités :\par
\pestelBullet Jardins ``réactivés'' : biodiversité locale.\par
\pestelBullet Sensibilisation à des pratiques plus durables.\par
\vspace{4pt}
Contraintes :\par
\pestelBullet Ne pas sur-promettre l’impact.\par
\pestelBullet Définir des indicateurs simples (heures, interventions).%
}
\hfill
\pestelBox{L — Légal}{%
Opportunités :\par
\pestelBullet RGPD : cadre clair pour protéger les données.\par
\pestelBullet CGU possibles pour poser des règles simples.\par
\vspace{4pt}
Contraintes :\par
\pestelBullet Données perso : limiter ce qui est public.\par
\pestelBullet Responsabilité civile à cadrer (incidents, dégâts).%
}

\vspace{3pt}
\caption{Analyse PESTEL - opportunités et contraintes du projet.}
\label{fig:js-pestel-synthese}
\end{figure}


\newpage

\subsection*{Benchmark}
\addcontentsline{toc}{subsection}{Benchmark}

Pour construire JardinSolidaire sur du concret, j’ai réalisé un \textbf{benchmark} : j’ai passé en revue plusieurs plateformes proches (mise en relation, entraide, services entre particuliers). Je me suis concentrée sur trois moments décisifs : \textbf{comment on trouve}, \textbf{comment on se met d’accord}, et \textbf{ce qui met en confiance} juste avant de s’engager.

Ce benchmark a guidé mes choix produit, avec comme objectif de réduire les hésitations et donner envie de passer du "je regarde" au "j’y vais".
\vspace{0.5cm}

\begin{figure}[H]
\centering

\newcolumntype{P}[1]{>{\raggedright\arraybackslash}p{#1}}
\newcolumntype{Y}{>{\raggedright\arraybackslash}X}

\renewcommand{\arraystretch}{1.45}
\setlength{\tabcolsep}{6pt} % évite de manger trop d’espace

\begin{tabularx}{\textwidth}{|P{2.6cm}|Y|Y|Y|Y|}
\hline
\textbf{Plateforme} & \textbf{Cible / promesse} & \textbf{Forces} & \textbf{Limites} & \textbf{Inspirations} \\
\hline
PlantezChezNous &
Co-jardinage / entraide &
Promesse alignée, esprit communauté &
Cadre opérationnel et sécurité à expliciter &
Promesse simple, règles claires, profils rassurants \\
\hline
AlloVoisins &
Services entre voisins &
Profils/avis, usage local &
Tendance prestation souvent payante &
Reprendre les codes de confiance, sans logique marchande \\
\hline
JardinPrivé.fr &
Accès à des jardins (location) &
Offre visible, accès à des lieux &
Modèle payant, peu d’entraide &
Se différencier : échange de temps + entraide \\
\hline
Privateaser &
Location d’espaces (événements) &
Fiches claires, critères lisibles &
Pas centré jardinage / entraide &
Fiches très lisibles : photos, infos, règles \\
\hline
\end{tabularx}

\normalsize

\normalsize
\vspace{6pt}
\caption{Benchmark - comparaison de solutions proches et enseignements appliqués.}
\label{fig:js-benchmark-synthese}
\end{figure}

\newpage


% ---------------------------
% Personas
% ---------------------------
\section*{Personas : profils utilisateur.ices}
\addcontentsline{toc}{section}{Personas : profils utilisateurs}

À partir des retours du questionnaire, j’ai construit des \textbf{personas} pour garder un repère concret pendant toute la conception. L’idée était simple : ne pas concevoir pour un.e utilisateur.ices "abstrait", mais pour des personnes réalistes, avec des contraintes et des attentes très différentes.

Un \textbf{persona} est un profil fictif, mais basé sur des éléments observés (motivations, freins, habitudes).

\subsection*{Personas}
\addcontentsline{toc}{subsection}{Personas}

\begin{figure}[H]
  \centering
  \includegraphics[width=\textwidth]{assets/03_specifications_fonctionnelles/personas.png}
  \caption{Personas}
  \label{fig:js-personas}
\end{figure}

\textbf{Louise, 45 ans} représente un profil de propriétaire avec un emploi du temps chargé. Elle veut un jardin entretenu sans devoir organiser toute une logistique. Son frein principal est la confiance : elle a besoin de savoir qui vient chez elle, et souhaite pouvoir consulter des profils clairs et des avis.

\textbf{Thomas, 28 ans} représente un jardinier volontaire qui vit en ville. Il veut jardiner pour apprendre et passer du temps dehors. Il craint de ne pas avoir assez d’expérience et de ne pas trouver de jardin près de chez lui. Il attend donc une recherche simple, idéalement avec une carte et des repères de localisation.

\textbf{Marie, 67 ans et Jean, 70 ans} représentent un couple de propriétaires retraités. Ils ont un grand jardin qu’ils entretenaient depuis des années, mais qui devient difficile à gérer physiquement. Ils souhaitent transmettre leur savoir, mais sont moins à l’aise avec les outils numériques. Leur attente principale est une interface simple, rassurante et intuitive, notamment pour gérer les créneaux et communiquer.

\newpage


% ---------------------------
% Objectifs fonctionnels
% ---------------------------
\section*{Objectifs fonctionnels majeurs}
\addcontentsline{toc}{section}{Objectifs fonctionnels majeurs}

À partir de ces besoins, j’ai défini les \textbf{objectifs fonctionnels majeurs}, c’est-à-dire les grandes actions que l’application doit permettre.

\subsection*{Simplicité d’accès}
\addcontentsline{toc}{subsection}{Simplicité d’accès}
L’application doit être utilisable depuis un navigateur, sur mobile comme sur ordinateur, et permettre d’accéder rapidement aux actions principales.

\subsection*{Publication des jardins}
\addcontentsline{toc}{subsection}{Publication des jardins}
Un propriétaire doit pouvoir décrire son jardin de façon claire et utile, avec des photos, une localisation approximative, et des informations qui permettent de comprendre ce qui est attendu.

\subsection*{Découverte et mise en relation}
\addcontentsline{toc}{subsection}{Découverte et mise en relation}
Un jardinier doit pouvoir explorer les jardins, filtrer selon des critères simples, puis envoyer une demande.

\subsection*{Organisation des interventions}
\addcontentsline{toc}{subsection}{Organisation des interventions}
L’application doit proposer un calendrier pour afficher et réserver des créneaux, ainsi qu’une messagerie pour préciser les consignes.

Il est aussi important que l’utilisateur comprenne facilement l’état de sa demande : "en attente", "acceptée", "planifiée", "réalisée", "annulée".
\vspace{0.3cm}

\begin{figure}[H]
  \centering
  \includegraphics[width=\textwidth]{assets/03_specifications_fonctionnelles/user_flow.png}
  \caption{Parcours principal - de la découverte à l’organisation de l’intervention.}
  \label{fig:js-user-flow}
\end{figure}

\newpage


% ---------------------------
% MVP
% ---------------------------
\section*{Fonctionnalités clés (MVP)}
\addcontentsline{toc}{section}{Fonctionnalités clés (MVP)}

À partir des besoins identifiés et des personas, j’ai défini le MVP (\textit{Minimum Viable Product}), c’est-à-dire la \textbf{plus petite version} du produit qui permet de \textbf{tester l’idée sur le terrain}.

Pour prioriser ce MVP, j’ai utilisé la méthode \textbf{MoSCoW}. Cette méthode consiste à classer les fonctionnalités en trois niveaux :
\textbf{Must-have} (indispensables), \textbf{Should-have} (importantes mais non indispensables), et \textbf{Could-have} (optionnelles, à ajouter plus tard).

Les \textbf{Must-have} correspondent au parcours principal complet : créer un compte, publier ou consulter un jardin, choisir un créneau, envoyer une demande, puis suivre son statut. Tout le reste peut venir ensuite.

Les \textbf{Should-have} regroupent ce qui améliore fortement l’expérience et la confiance, sans être nécessaire pour valider le cœur du produit : par exemple une recherche plus avancée, une carte, un calendrier plus interactif, des avis, ou une connexion via des services tiers.

Enfin, les \textbf{Could-have} correspondent aux ajouts qui enrichissent la communauté une fois que le parcours principal est solide : une banque du temps (cumul d’heures), des badges, ou des récits d’expérience.

L’objectif de cette méthode est donc d’obtenir un produit cohérent, complet et testable dès maintenant, puis de l’enrichir progressivement sans se disperser.

\begin{figure}[H]
  \centering
  \includegraphics[width=0.95\textwidth]{assets/03_specifications_fonctionnelles/moscow.png}
  \caption{Priorisation MoSCoW des fonctionnalités (MVP et évolutions).}
  \label{fig:js-moscow}
\end{figure}

\newpage

\subsection*{Création de compte}
\addcontentsline{toc}{subsection}{Création de compte}
Inscription par e-mail et mot de passe, puis validation du compte via un code reçu par e-mail. Les informations du profil (photo, rôle, localisation, besoins ou compétences) sont consultables et modifiables par l’utilisateur.

\subsection*{Connexion}
\addcontentsline{toc}{subsection}{Connexion}
Authentification pour accéder aux fonctionnalités principales (publication, recherche, calendrier, messagerie) et une double authentification (2FA) par code e-mail.

\subsection*{Publication et gestion des jardins (propriétaires)}
\addcontentsline{toc}{subsection}{Publication et gestion des jardins (propriétaires)}
Création et édition d’une fiche jardin avec description, photos, localisation (approximative), superficie, besoins (tonte, désherbage, arrosage, plantation, etc.) et créneaux disponibles. L’utilisateur contrôle la visibilité de son jardin.

\subsection*{Recherche de jardins (jardinier·es)}
\addcontentsline{toc}{subsection}{Recherche de jardins (jardinier·es)}
Exploration des jardins via une liste et des filtres simples (activité, disponibilités, rayon), avec la possibilité d’ajouter des favoris et d’accéder à des fiches détaillées (photos, consignes, outils disponibles).

\subsection*{Carte}
\addcontentsline{toc}{subsection}{Carte}
Affichage des jardins sur une carte, en complément de la liste, pour faciliter la recherche à proximité.

\subsection*{Demandes d’aide et réservations}
\addcontentsline{toc}{subsection}{Demandes d’aide et réservations}
Envoi d’une demande sur un créneau libre. Suivi d’un statut clair : \textit{en attente}, \textit{acceptée}, \textit{refusée}, \textit{planifiée}, \textit{réalisée} ou \textit{annulée}. Le propriétaire peut accepter ou refuser, et le jardinier peut annuler avant la réalisation.

\subsection*{Messagerie intégrée}
\addcontentsline{toc}{subsection}{Messagerie intégrée}
Discussion liée à un jardin ou à une réservation, afin de clarifier les consignes (accès, outils, durée, point de rencontre). Des notifications par e-mail sont envoyées en cas de nouveau message.

\subsection*{Tableau de bord et historique}
\addcontentsline{toc}{subsection}{Tableau de bord et historique}
Un espace de suivi permet de retrouver les demandes envoyées (côté jardinier·es) ou reçues (côté propriétaire), ainsi que l’historique des interventions à venir et passées.

\subsection*{Extensions prévues (post-MVP)}
\addcontentsline{toc}{subsection}{Extensions prévues (post-MVP)}
Une fois le MVP stable, plusieurs extensions sont envisagées : connexion via Google, système complet d’avis et de notation, notifications en temps réel, cumul d’heures (banque du temps), badges et éléments de gamification.
